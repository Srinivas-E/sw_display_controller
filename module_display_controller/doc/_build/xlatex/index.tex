% Generated by Sphinx.
\def\sphinxdocclass[english]{xmosmodern}
\documentclass[  collection]{xmosmodern}
\usepackage[utf8]{inputenc}
\DeclareUnicodeCharacter{00A0}{\nobreakspace}

\usepackage{longtable}



\title{Display Controller Component}
\date{September 19, 2013}
\author{}
\newcommand{\sphinxlogo}{}
\newcommand{\releasename}{Release}
\usepackage{xsphinx}
\usepackage{threeparttable}
\usepackage{fancyvrb}
\usepackage{indent}
\renewcommand\bfcode\textbf
\renewcommand\bf\textbf
\graphicspath{{./}{./images/}}
\makeindex

\newcommand\PYGZat{@}
\newcommand\PYGZlb{[}
\newcommand\PYGZrb{]}

\setlength{\emergencystretch}{8em}
\start

\maketitle
\pretoc
\phantomsection\label{index::doc}

%summary!

% NON-FULLWIDTH SECTION



% NON-FULLWIDTH SECTION
\clearpage
\chapter{Overview}
\label{overview:display-controller-component}\label{overview::doc}\label{overview:overview}%summary!
\begin{inthisdocument}
\item \nameref{overview:features}
\item \nameref{overview:memory-requirements}
\item \nameref{overview:resource-requirements}
\item \nameref{overview:performance}
\end{inthisdocument}



The display controller module is used to drive a single graphics LCD screen up to 800 * 600 pixels incorporating a managed double buffer.



% NON-FULLWIDTH SECTION
\section{Features}
\label{overview:features}\begin{itemize}
\item   Non-blocking SDRAM management.

\item   Real time servicing of the LCD.

\item   Touch interactive display

\item   Image memory manager to simplify handling of images.

\item   No real time constraints on the application.

\end{itemize}




% NON-FULLWIDTH SECTION
\section{Memory requirements}
\label{overview:memory-requirements}
\begin{tabular}{ll}
\Toprule
\textbf{Resource} & \textbf{Usage}\\
\midrule
Stack & 6198 bytes\\
Program & 11306 bytes\\
\bottomrule
\end{tabular}




% NON-FULLWIDTH SECTION
\section{Resource requirements}
\label{overview:resource-requirements}
\begin{tabular}{ll}
\Toprule
\textbf{Resource} & \textbf{Usage}\\
\midrule
Channels & 3\\
Timers & 0\\
Clocks & 0\\
Threads & 1\\
\bottomrule
\end{tabular}




% NON-FULLWIDTH SECTION
\section{Performance}
\label{overview:performance}

The achievable effective bandwidth varies according to the avaliable XCore MIPS. The maximum pixel clock supported is 25MHz.



% NON-FULLWIDTH SECTION
\clearpage
\chapter{Hardware Requirements}
\label{hw::doc}\label{hw:evaluation-platforms}%summary!
\begin{inthisdocument}
\item \nameref{hw:sec-hardware-platforms}
\item \nameref{hw:demonstration-applications}
\end{inthisdocument}




% NON-FULLWIDTH SECTION
\section{Recommended Hardware}
\label{hw:sec-hardware-platforms}\label{hw:recommended-hardware}


% NON-FULLWIDTH SECTION
\subsection{Slicekit}
\label{hw:slicekit}

This module may be evaluated using the Slicekit Modular Development Platform, available from digikey. Required board SKUs are:
\begin{itemize}
\item   XP-SKC-L2 (Slicekit L2 Core Board)

\item   XA-SK-SCR480 plus XA-SK-XTAG2 (Slicekit XTAG adaptor)

\end{itemize}




% NON-FULLWIDTH SECTION
\section{Demonstration Applications}
\label{hw:demonstration-applications}


% NON-FULLWIDTH SECTION
\subsection{Display Controller Application}
\label{hw:display-controller-application}\begin{itemize}
\item   Package: sw\_display\_controller

\item   Application: app\_display\_controller

\end{itemize}



This combination demo employs the \verb`module_lcd` along with the \verb`module_sdram`, \verb`module_touch_controller_lib`, \verb`module_i2c_master` and the \verb`module_display_controller` framebuffer framework component to implement a 480x272 display controller.


Required board SKUs for this demo are:
\begin{itemize}
\item   XP-SKC-L2 (Slicekit L2 Core Board) plus XA-SK-XTAG2 (Slicekit XTAG adaptor)

\item   XA-SK-SDRAM

\item   XA-SK-SCR480 (which includes a 480x272 color touch screen)

\end{itemize}




% NON-FULLWIDTH SECTION
\clearpage
\chapter{API}
\label{api:sec-display-controller-api}\label{api::doc}\label{api:project-structure}%summary!
\begin{inthisdocument}
\item \nameref{api:configuration-defines}
\item \nameref{api:api}
\end{inthisdocument}

\begin{description}
\item[To build a project including the \ttfamily module\_display\_controller the following components are required:]\begin{itemize}
\item   component: sc\_sdram\_burst which handles the SDRAM

\item   component: sc\_lcd which handles the LCD

\end{itemize}


\end{description}



The below section details the APIs in the application. For details about the LCD and SDRAM APIs please refer to the respective repositories.



% NON-FULLWIDTH SECTION
\section{Configuration Defines}
\label{api:configuration-defines}

The \verb`module_display_controller` can be configured via the header \verb`display_controller_conf.h`. The module requires nothing to be additionally defined however any of the defines can be overridden by adding the header \verb`display_controller_conf.h` to the application project and adding the define that needs overridding. The possible defines are:
\begin{description}
\item[\textbf{DISPLAY\_CONTROLLER\_MAX\_IMAGES}]

This defines the storage space allocated to the display controller for it to store image metadata. When an image is registered with the display controller its dimensions and location in SDRAM address space are stored in a table. The define specifies how many entries are allowed in that table. Note, there is no overflow checking by default.

\item[\textbf{DISPLAY\_CONTROLLER\_VERBOSE}]

This define switches on the error checking for memory overflows and causes verbose error warnings to be emitted in the event of an error.

\end{description}




% NON-FULLWIDTH SECTION
\section{API}
\label{api:api}\begin{description}
\item[The{}`{}`module\_display\_controller{}`{}` functionality is defined in]\begin{itemize}
\item   \verb`display_controller_client.xc`

\item   \verb`display_controller_internal.h`

\item   \verb`display_controller.xc`

\item   \verb`display_controller.h`

\item   \verb`transitions.h`

\item   \verb`transitions.xc`

\end{itemize}


\end{description}



The display controller handles the double buffering of the image data to the LCD as a real time service and manages the I/O to the SDRAM as a non-real time service.


The display controller API is as follows:
.. doxygenfunction:: display\_controller
.. doxygenfunction:: display\_controller\_image\_read\_line
.. doxygenfunction:: display\_controller\_image\_read\_line\_p
.. doxygenfunction:: display\_controller\_image\_write\_line
.. doxygenfunction:: display\_controller\_image\_write\_line\_p
.. doxygenfunction:: display\_controller\_image\_read\_partial\_line
.. doxygenfunction:: display\_controller\_image\_read\_partial\_line\_p
.. doxygenfunction:: display\_controller\_register\_image
.. doxygenfunction:: display\_controller\_wait\_until\_idle
.. doxygenfunction:: display\_controller\_wait\_until\_idle\_p
.. doxygenfunction:: display\_controller\_frame\_buffer\_commit
.. doxygenfunction:: display\_controller\_frame\_buffer\_init


The transition API is as follows:
.. doxygenfunction:: transition\_wipe
.. doxygenfunction:: transition\_slide
.. doxygenfunction:: transition\_roll
.. doxygenfunction:: transition\_dither
.. doxygenfunction:: transition\_alpha\_blend


The transitions use the display controller API.



% NON-FULLWIDTH SECTION
\clearpage
\chapter{Programming Guide}
\label{programming:programming-guide}\label{programming::doc}%summary!
\begin{inthisdocument}
\item \nameref{programming:shared-memory-interface}
\item \nameref{programming:source-code-structure}
\item \nameref{programming:executing-the-project}
\item \nameref{programming:software-requirements}
\end{inthisdocument}




% NON-FULLWIDTH SECTION
\section{Shared Memory Interface}
\label{programming:shared-memory-interface}

The display controller uses a shared memory interface to move the large amount of data around from tile to tile efficiently. This means that the \verb`display_controller`, \verb`sdram_server` and \verb`lcd_server` must be one the same tile.



% NON-FULLWIDTH SECTION
\section{Source code structure}
\label{programming:source-code-structure}
\begin{figure}[h]\begin{sidecaption}{Project structure}
\small
\begin{tabularx}{\linewidth}{llY}
\Toprule
\textbf{Project} & \textbf{File} & \textbf{Description}\\
\midrule
module\_display\_controller & \verb`display_controller.h` & Header file containing the APIs for the display controller component.\\
 & \verb`display_controller.xc` & File containing the implementation of the display controller component.\\
 & \verb`display_controller_client.xc` & File containing the implementation of the display controller client functions.\\
 & \verb`display_controller_internal.h` & Header file containing the user configurable defines for the display controller component.\\
 & \verb`transitions.h` & Header file containing the APIs for the display controller transitions.\\
 & \verb`transitions.xc` & File containing the implementation of the display controller transitions.\\
\bottomrule
\end{tabularx}

\end{sidecaption}
\end{figure} \DocumentFooterFix



% NON-FULLWIDTH SECTION
\section{Executing The Project}
\label{programming:executing-the-project}

The module by itself cannot be built or executed separately - it must be linked in to an application. Once the module is linked to the application, the application can be built and tested for driving a LCD screen.
\begin{description}
\item[The following modules should be added to the list of MODULES in order to link the component to the application project]\begin{enumerate}
\item   \verb`module_display_controller`

\item   \verb`module_lcd`

\item   \verb`module_sdram`

\end{enumerate}


\item[Now the module is linked to the application and can be used directly. Additionally, if the use of the touch controller is nessessary then]\begin{enumerate}
\item   \verb`module_touch_controller_lib` or \verb`module_touch_controller_server`

\item   \verb`module_i2c_master`

\end{enumerate}


\end{description}



should be added to the list of MODULES.



% NON-FULLWIDTH SECTION
\section{Software Requirements}
\label{programming:software-requirements}

The module is built on XDE Tool version 12.0
The module can be used in version 12.0 or any higher version of xTIMEcomposer.



% NON-FULLWIDTH SECTION
\clearpage
\chapter{Example Applications}
\label{examples:example-application}\label{examples::doc}%last summary
\begin{inthisdocument}
\item \nameref{examples:app-display-controller-demo}
\item \nameref{examples:application-notes}
\end{inthisdocument}



This tutorial describes a demo application that uses the display controller module. \Sec~\ref{hw:sec-hardware-platforms} describes the required hardware setup to run the demos.



% NON-FULLWIDTH SECTION
\section{app\_display\_controller\_demo}
\label{examples:app-display-controller-demo}

This application demonstrates how the \verb`lcd_module` is used to write image data to the LCD screen whilst imposing no real time constraints on the application. The purpose of this demonstration is to show how data is passed to the \verb`display_controller`. This application also demonstrates an interactive display using \verb`touch_controller_lib` module.



% NON-FULLWIDTH SECTION
\section{Application Notes}
\label{examples:application-notes}


% NON-FULLWIDTH SECTION
\subsection{Getting Started}
\label{examples:getting-started}\begin{enumerate}
\item   Plug the XA-SK-LCD Slice Card into the `TRIANGLE' slot of the Slicekit Core Board

\item   Plug the XA-SK-SDRAM Slice Card into the `STAR' slot of the Slicekit Core Board

\item   Open \verb`app_display_controller_demo.xc` and build the project.

\item   run the program ensuring that it is run from the project directory where the tga images are.

\end{enumerate}



The output produced should look like a series of images transitioning on the LCD when the screen is touched.



\posttoc

\finish
